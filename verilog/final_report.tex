\documentclass{jarticle}
                                \usepackage[dvipdfmx]{graphicx}
\author{21班 新藤光 1029-26-4887}
\date{\today}
\title{ハードウェア実験3最終レポートA}
\begin{document}

\maketitle

\section{基本仕様}
\label{sec-1}
\subsection{概要}
\label{sec-1-1}
今回作成したアーキテクチャの名前を以下SIMPLE-J とする。
SIMPLE-Jは16bitの命令を実行する簡単なアーキテクチャである。
\subsection{アーキテクチャ}
\label{sec-1-2}
\subsubsection{主記憶とレジスタ}
\label{sec-1-2-1}
主記憶とレジスタは、それぞれすべて16bit幅である。
\begin{itemize}
\item 主記憶
\end{itemize}
16bit幅であり、命令の格納とデータの格納・読み出しは同じ主記憶上で行う。
\begin{itemize}
\item レジスタ
\end{itemize}
16bit幅のレジスタt0\textasciitilde{}t7がある。
\subsection{特徴}
\label{sec-1-3}
シングルコアであり、かつ1命令を5つのフェーズ1からフェーズ5に分け、フェーズ1とフェーズ3、フェーズ2とフェーズ5をそれぞれ並行に処理する。
\subsection{命令セット}
\label{sec-1-4}
\subsubsection{命令形式}
\label{sec-1-4-1}
SIMPLE-Jの命令はすべて16ビットの固定長である。
以下に命令セットを示す。
\begin{enumerate}
\item 算術演算\\
\label{sec-1-4-1-1}
レジスタRdとRsの結果の加算、減算をRdに格納する。
\item 論理演算\\
\label{sec-1-4-1-2}
レジスタRdとRsのAnd、Or、XORの結果をRdに格納する。
\item 移動演算\\
\label{sec-1-4-1-3}
レジスタRdにRsの値を格納する。
\item シフト演算\\
\label{sec-1-4-1-4}
左論理シフト、左循環シフト、右論理シフト、右算術シフトの結果をRdに格納する。
\item 入出力命令\\
\label{sec-1-4-1-5}
\begin{itemize}
\item OUT レジスタRsの値を出力機器に出力する。
\item HLT 動作を停止させる。
\end{itemize}
\item ロードストア命令\\
\label{sec-1-4-1-6}
メモリに値を書き込んだり値を読みだしたりする。
\item 条件分岐命令\\
\label{sec-1-4-1-7}
分岐条件が成り立つときPCを書きかえて分岐する。
\begin{center}
\begin{tabular}{rllll}
\hline
15   14 & 13  11 & 10  8 & 7    4 & 3  0\\
\hline
11 & Rs & Rd & op3 & d\\
\hline
\end{tabular}
\end{center}

\begin{center}
\begin{tabular}{lrl}
\hline
mnemonic & op3 & function\\
\hline
ADD Rd,Rs & 0000 & r[Rd] = r[Rd] + r[Rs]\\
\hline
SUB Rd,Rs & 0001 & r[Rd] = r[Rd] - r[Rs]\\
\hline
AND Rd,Rs & 0010 & r[Rd] = r[Rd] \& r[Rs]\\
\hline
OR Rd,Rs & 0011 & r[Rd] = r[Rd]  r[Rs]\\
\hline
XOR Rd,Rs & 0100 & r[Rd] = r[Rd] \^{} r[Rs]\\
\hline
CMP Rd,Rs & 0101 & r[Rd] -  r[Rs]\\
\hline
MOV Rd,Rs & 0110 & r[Rd] =  r[Rs]\\
\hline
SLL Rd,Rs & 1000 & r[Rd] = sll(r[Rd], d)\\
\hline
SLR Rd,Rs & 0011 & r[Rd] = slr(r[Rd], d)\\
\hline
SRL Rd,Rs & 1001 & r[Rd] = srl(r[Rd], d)\\
\hline
SRA Rd,Rs & 1011 & r[Rd] = sra(r[Rd], d)\\
\hline
OUT Rs & 1101 & output = r[Rs]\\
\hline
HLT & 1111 & halt()\\
\hline
\end{tabular}
\end{center}


\begin{center}
\begin{tabular}{llll}
\hline
15   14 & 13  11 & 10  8 & 7       0\\
\hline
op1 & Ra & Rb & d\\
\hline
\end{tabular}
\end{center}

\begin{center}
\begin{tabular}{lrl}
\hline
mnemonic & op1 & function\\
\hline
LI Rb,d & 000 & r[Rb] = sign$_{\text{ext}}$(d)\\
\hline
addi Rb & 001 & r[Rb] = r[Rb] + sign$_{\text{ext}}$(d)\\
\hline
subi Rb & 010 & r[Rb] = r[Rb] + sign$_{\text{ext}}$(d)\\
\hline
B Rb,d & 100 & PC = PC + 1 + sign$_{\text{ext}}$(d)\\
\hline
\end{tabular}
\end{center}


\begin{center}
\begin{tabular}{rrll}
\hline
15 14 & 13 11 & 10 8 & 7 0\\
\hline
10 & 111 & cond & d\\
\hline
\end{tabular}
\end{center}

\begin{center}
\begin{tabular}{lrl}
\hline
mnemonic & cond & function\\
\hline
BE d & 000 & if (Z) PC = PC + 1 + sign$_{\text{ext}}$(d)\\
\hline
BLT d & 001 & if (S \^{} V) PC = PC + 1 + sign$_{\text{ext}}$(d)\\
\hline
BLE d & 010 & if (Z ∨ (S \^{} V)) PC = PC + 1 + sign$_{\text{ext}}$(d)\\
\hline
BNE d & 011 & if (!Z) PC = PC + 1 + sign$_{\text{ext}}$(d)\\
\hline
\end{tabular}
\end{center}
\end{enumerate}


\subsection{基本的な設計}
\label{sec-1-5}
\subsubsection{制御回路}
\label{sec-1-5-1}
\begin{enumerate}
\item clock\\
\label{sec-1-5-1-1}
適切な発振回路を用いてクロックを提供する。クロックの立ち上がりによって各フェーズを順に活性化する。
\item reset\\
\label{sec-1-5-1-2}
reset信号の立ち上がりによって、回路内のすべてのDFFの値を0に初期化する。
\end{enumerate}

\section{SIMPLEBの基本仕様からの拡張および性能評価}
\label{sec-2}
\subsection{基本アーキテクチャの拡張}
\label{sec-2-1}
\subsubsection{命令の拡張}
\label{sec-2-1-1}
\begin{enumerate}
\item 即値演算\\
\label{sec-2-1-1-1}
即値を足したり引いたりするとき、無駄なロード命令、ストア命令を減らすことができるようになる。
\item フェーズ1とフェーズ3、フェーズ2とフェーズ5の同時実行。\\
\label{sec-2-1-1-2}
\end{enumerate}

\subsection{プロセッサの性能評価}
\label{sec-2-2}
\subsubsection{ゲート数 1027/5980(17\%)}
\label{sec-2-2-1}
\subsubsection{Fmax 57.32MHz}
\label{sec-2-2-2}
\subsubsection{クリティカルパス}
\label{sec-2-2-3}
From rab:inst20|a-out3
To  ram:ram1
\subsubsection{拡張の評価}
\label{sec-2-2-4}
\begin{enumerate}
\item 並列化するにあたり必要な回路が増えたので、ゲート数は増えた。
\label{sec-2-2-4-1}
\end{enumerate}

\section{分担状況}
\label{sec-3}
すべて自分で設計しました。
% Emacs 24.5.1 (Org mode 8.2.10)
\end{document}
